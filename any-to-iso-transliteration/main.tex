\documentclass[10pt]{article}

% This assumes your files are encoded as UTF8
\usepackage[utf8]{inputenc}

% Devanagari Related Packages
\usepackage{fontspec, xunicode, xltxtra}

% Define Fonts
\newfontfamily\textdevanagari[Script=Devanagari]{Noto Serif Devanagari}
\newfontfamily\textbengali[Script=Bengali]{Noto Serif Bengali}
\newfontfamily\texttelugu[Script=Telugu]{Noto Serif Telugu}
\newfontfamily\textkannada[Script=Kannada]{Noto Serif Kannada}

\newfontfamily\textiso{Noto Serif}

% Commands for Indian Language Transliterations
\newcommand{\devanagari}[1]{{\textdevanagari{#1}}}
\newcommand{\bengali}[1]{{\textbengali{#1}}}
\newcommand{\telugu}[1]{{\texttelugu{#1}}}
\newcommand{\kannada}[1]{{\textkannada{#1}}}

\newcommand{\iso}[1]{{\textiso{#1}}}
\newcommand{\Iso}[1]{{\textiso{#1}}}
\newcommand{\ISO}[1]{{\textiso{#1}}}

\title{Transliteration of Indian Language Scripts to ISO}

\author{Hrishikesh Terdalkar}

\begin{document}

\maketitle

\section{Introduction}

We provide this setup as a minimal example to transliterate from any script (that is supported by
the \texttt{indic\_transliteration} package) to ISO transliteration scheme.

The \texttt{\textbackslash devanagari}, \texttt{\textbackslash bengali}, \texttt{\textbackslash telugu}
and \texttt{\textbackslash kannada} commands and their respective font families are used in this document to render the
text in the original scripts. They are not required if the text is not to be rendered in the original scripts.
Commands like \texttt{\textbackslash iso}, \texttt{\textbackslash Iso} and
\texttt{\textbackslash ISO} can be used to enclose text in any script supported by \texttt{indic\_transliteration}
and the ISO version of the text will be generated using \texttt{finalize.py}.


\section{Transliteration}

We now showcase examples using various scripts.

\subsection{Devanagari}

\noindent\texttt{\textbackslash devanagari\{\devanagari{को न्वस्मिन् साम्प्रतं लोके गुणवान् कश्च वीर्यवान्।}\}}\\
\noindent\texttt{\textbackslash iso\{\devanagari{को न्वस्मिन् साम्प्रतं लोके गुणवान् कश्च वीर्यवान्।}\}}\\
\noindent\texttt{\textbackslash Iso\{\devanagari{को न्वस्मिन् साम्प्रतं लोके गुणवान् कश्च वीर्यवान्।}\}}\\
\noindent\texttt{\textbackslash ISO\{\devanagari{को न्वस्मिन् साम्प्रतं लोके गुणवान् कश्च वीर्यवान्।}\}}\\

\noindent \textbf{Output}:\\

\noindent\devanagari{को न्वस्मिन् साम्प्रतं लोके गुणवान् कश्च वीर्यवान्।}\\
\noindent\iso{को न्वस्मिन् साम्प्रतं लोके गुणवान् कश्च वीर्यवान्।}\\
\noindent\Iso{को न्वस्मिन् साम्प्रतं लोके गुणवान् कश्च वीर्यवान्।}\\
\noindent\ISO{को न्वस्मिन् साम्प्रतं लोके गुणवान् कश्च वीर्यवान्।}

\subsection{Bengali}

\noindent\texttt{\textbackslash bengali\{\bengali{বন্দে মাতরম}\}}\\
\noindent\texttt{\textbackslash iso\{\bengali{বন্দে মাতরম}\}}\\
\noindent\texttt{\textbackslash Iso\{\bengali{বন্দে মাতরম}\}}\\
\noindent\texttt{\textbackslash ISO\{\bengali{বন্দে মাতরম}\}}\\

\noindent \textbf{Output}:\\

\noindent\bengali{বন্দে মাতরম}\\
\noindent\iso{বন্দে মাতরম}\\
\noindent\Iso{বন্দে মাতরম}\\
\noindent\ISO{বন্দে মাতরম}

\subsection{Kannada}

\noindent\texttt{\textbackslash kannada\{\kannada{ಓಂ ತ್ರಯಂಬಕಂ ಯಜಾಮಹೇ}\}}\\
\noindent\texttt{\textbackslash iso\{\kannada{ಓಂ ತ್ರಯಂಬಕಂ ಯಜಾಮಹೇ}\}}\\
\noindent\texttt{\textbackslash Iso\{\kannada{ಓಂ ತ್ರಯಂಬಕಂ ಯಜಾಮಹೇ}\}}\\
\noindent\texttt{\textbackslash ISO\{\kannada{ಓಂ ತ್ರಯಂಬಕಂ ಯಜಾಮಹೇ}\}}\\

\noindent \textbf{Output}:\\

\noindent\kannada{ಓಂ ತ್ರಯಂಬಕಂ ಯಜಾಮಹೇ}\\
\noindent\iso{ಓಂ ತ್ರಯಂಬಕಂ ಯಜಾಮಹೇ}\\
\noindent\Iso{ಓಂ ತ್ರಯಂಬಕಂ ಯಜಾಮಹೇ}\\
\noindent\ISO{ಓಂ ತ್ರಯಂಬಕಂ ಯಜಾಮಹೇ}\\

\subsection{Telugu}

\noindent\texttt{\textbackslash telugu\{\telugu{కాశికా పురాధినాథ కాలభైరవం భజే}\}}\\
\noindent\texttt{\textbackslash iso\{\telugu{కాశికా పురాధినాథ కాలభైరవం భజే}\}}\\
\noindent\texttt{\textbackslash Iso\{\telugu{కాశికా పురాధినాథ కాలభైరవం భజే}\}}\\
\noindent\texttt{\textbackslash ISO\{\telugu{కాశికా పురాధినాథ కాలభైరవం భజే}\}}\\

\noindent \textbf{Output}:\\

\noindent\telugu{కాశికా పురాధినాథ కాలభైరవం భజే}\\
\noindent\iso{కాశికా పురాధినాథ కాలభైరవం భజే}\\
\noindent\Iso{కాశికా పురాధినాథ కాలభైరవం భజే}\\
\noindent\ISO{కాశికా పురాధినాథ కాలభైరవం భజే}


\section{Summary}

The procedure to transliterate text from any script is identical.
\texttt{\textbackslash iso}, \texttt{\textbackslash Iso} and
\texttt{\textbackslash ISO} commands will work for any script, detecting it
automatically prior to transliteration.


\pagebreak

% include your own bib file like this
\nocite{*}
\bibliographystyle{acm}
\bibliography{papers}

% --------------------------------

\end{document}

